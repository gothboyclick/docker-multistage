\documentclass{modelo_banner}

\primeiroAutor{Adriano Xavier Nobre Junior}
\abreviacaoPrimeiroAutor{NOBRE JUNIOR, A. X.}
\notaPrimeiroAutor{Estudante do curso de bacharelado em Sistemas de Informação, IF Goiano – Campus Ceres.}

\segundoAutor{Ronneesley Moura Teles}
\abreviacaoSegundoAutor{TELES, R. M.}
\notaSegundoAutor{Professor orientador, IF Goiano – Campus Ceres.}

\titulo{Docker Multistage na prática}

\begin{document}

	\construirtitulo

	\construirautores

	\begin{multicols}{2}	
		\section{Introdução}
		Durante o desenvolvimento de projetos de aplicações é muito comum que se encontre problemas e conflitos de versões de aplicações e sistemas operacionais dos diferentes desenvolvedores envolvidos na construção do ambiente, durante o desenvolvimento do projeto do SLab que é um software de ferramenta estatística, se enfrentou esse problema, isso causou o surgimento de alternativas conhecidas como Conteinerização que é muito comum sua utilização por serem consideradas leves, rápidas e balanceadas, a ferramenta mais famosa e para pronto uso é o \textit{Docker}, esse Docker serve para fazer essa separação de recursos utilizados no desenvolvimento e revisando a documentação, os desenvolvedores identificaram uma opção de usar diferentes ferramentas em apenas um arquivo docker, diminuindo, em teoria, o tamanho da imagem de base que esse \textit{container} vai usar. Deve-se observar que essa prática vai de fato refletir e se a mesma poderá ser construída de diferentes formas.
		\section{Materiais e Métodos}
		Usando as recomendações da documentação do docker buscou-se entender o conceito de \textit{docker} \textit{multistage} para implementá-lo no projeto, durante a criação da imagem que o \textit{container} vai usar como base para rodar a aplicação, poderemos observar a diferença de tamanhos geradas na criação com o \textit{docker} \textit{multistage} e a imagem gerada em um pequeno servidor Ubuntu 22.04, adicionando as dependências das aplicações.
	Para isso vai se precisar de um computador/notebook com capacidade de virtualização, conhecida como \textit{hypervisor}, com \textit{docker} na versão 24 instalado, será criado os aquivos para criação das máquinas e com as respectivas configurações para cada ferramenta, após isso será realizado o comando de execução do \textit{docker} para cada um dos arquivos e comparado o tamanho final de cada uma das imagens geradas com as dependências preenchidas, nesse caso usamos NodeJS na versão 18, PHP na versão 8.1 e Composer na versão 2.5.
		
		\section{Resultados}
			 Como resultados foi observado que a utilização do \textit{multistage} dentro de projetos gerou uma imagem *bem menor*, diferente do que foi executado separado onde as dependências eram criadas e mantidas dentro do repositório local de imagens, isso abre porta para uma diminuição dos recursos de armazenamentos em \textit{Runners} de \textit{pipelines} promovendo a redução de valores de armazenamentos em projetos reais que no geral usam esses \textit{runners} para manter.
		\section{Considerações finais}
			 Concluindo podemos promover a divulgação do recurso do docker promovendo a visibilidade e redução de custos de armazenameto, abrindo possibilidades para outras otimizações dentro do ambiente docker.
		Pontos Importantes para serem considerados:
		
		\begin{itemize}
			\item \textit{Docker} consegue manter-se atualizada frente as necessidades do mercado;
			
			\item Foi possível otimizar os recursos dos \textit{runners} do projeto que usariam maiores recursos para construir uma nova imagem da aplicação;
		\end{itemize}		
		\section{Agradecimentos}
		
		Nesta seção são feitos os agradecimentos a quem considerar importante.
		
		\referencias		
		https://docs.docker.com/
	\end{multicols}
	
	\rodape

\end{document}