\documentclass{modelo_resumo_simples}

\primeiroAutor{Adriano Xavier Nobre Junior}
\abreviacaoPrimeiroAutor{NOBRE JUNIOR, A. X.}
\notaPrimeiroAutor{Estudante do curso de bacharelado em Sistemas de Informação, IF Goiano – Campus Ceres.}

\segundoAutor{Ronneesley Moura Teles}
\abreviacaoSegundoAutor{TELES, R. M.}
\notaSegundoAutor{{Professor orientador, IF Goiano – Campus Ceres.}}
\titulo{\textit{Docker Multistage} na prática}

\begin{document}

	\construirtitulo

	\construirautores
	
	\begin{resumo}	
	Durante o desenvolvimento de projetos de aplicações é muito comum que se encontre problemas e conflitos de versões de aplicações e sistemas operacionais isso causou o surgimento de Máquinas Virtuais e alternativas conhecidas como Conteinerização que é muito comum sua utilização por serem consideradas leves, rápidas e balanceadas, a ferramenta mais famosa e para pronto uso é o \textit{Docker}. 
	Mas de fato se obtém imagens leves e rápidas dentro de projetos com complexidade moderada a alta ? O presente trabalho apresentará diferentes possibilidades de aplicação dessa ferramenta no cenário de projetos que utilizam a Conteinerização para gerenciar as ferramentas de desenvolvimento. 
	Para isso vamos precisar de um computador/notebook com capacidade de virtualização, conhecida como \textit{hypervisor}, com \textit{docker} instalado, criamos os aquivos para criação das máquinas e com as configurações, após isso fizemos o comando de execução do \textit{docker} para cada um dos arquivos e comparamos o tamanho final de cada uma das imagens geradas com as dependências preenchidas. Como resultados observamos que a utilização do \textit{multistage} dentro de projetos gerou uma imagem bem menor, diferente do que foi feito separado onde as dependências eram criadas e mantidas dentro do repositório local de imagens, isso abre porta para uma diminuição dos recursos de armazenamentos em \textit{Runners} de pipelines promovendo a redução de valores em projetos reais. Concluindo podemos promover a divulgação do recurso do \textit{docker} promovendo a visibilidade e redução de custos, abrindo possibilidades para outras otimizações dentro do ambiente \textit{docker}.
	\end{resumo}
	
	\begin{palavras_chave}
	Computação; Algoritmo; \textit{Infrastructure as a Code}
	\end{palavras_chave}

\end{document}